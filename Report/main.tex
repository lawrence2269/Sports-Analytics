\documentclass[format=sigconf]{acmart}
\settopmatter{printacmref=false} % Removes citation information below abstract
\renewcommand\footnotetextcopyrightpermission[1]{} % removes footnote with conference information in first column
\usepackage[utf8]{inputenc}
\usepackage{amsmath}

\settopmatter{printacmref=false}

\title{Mathematical model approach for draft picking in basketball}

\author{Lawrence Thanakumar Rajappa}
\affiliation{\institution{IDA Linköping University}}
\email{lawra776@student.liu.se}

\date{December 2020}
\begin{document}
\maketitle
\pagestyle{plain} % Remove ACM page header

\section{Introduction}
Analytics is being used in all fields such as healthcare, manufacturing, banking and etc. for decision-making. Likewise, 
analytics is also playing a major role in Sports industry such as football, baseball, basketball and etc. to predict player's 
next move, injury analysis, position analysis and etc. Sports analytics is being spoken as a concept for many years which 
could be used to improve team performance, as a result, the revenue generation is very much improved for the team. For this 
project, we will mostly focus on basketball. In order to have a good prediction and analysis report, we need to have proper
dataset and most importantly, the data must also have most important attributes that could provide insights for a given problem.

There are many ways that the data can be used by the team for various purposes. The kind of data that the team would use includes
average stats of the players, per game stats, and etc. These data can be used to understand a player in terms of strengths and 
weaknesses, emotional stability and etc. These attributes could be used to assess the team's performance. Moreover, there are other 
attributes such as weather conditions, the condition of the field and even psychological factors such as the fans support should be
included along with player's data to determine the team's performance. This document speaks about how the data are being used by
a basketball team to select players using mathematical models.

\section{Aim}
In this project, the aim is to create a draft picking system for a basketball team by using 3 mathematical models namely, model 1 : model
to predict whether a player will stay in the team for five years or not, model 2: model to determine the position of players 
based on their previous experiences and model 3: model to cluster or group players based on previous performances. These models
would facilitate team managers and coaches to select players and make best out of them.

\section{Motivation}
Before the advent of analytics, the selection of players or draft formation was done manually which was a time consuming and huge 
workload. The emergence of analytics and computing resources has paved a new way in recruiting best players based on their previous
performances in a short period of time with minimal workload. However, Sports industry has restricted for the complete adoption of 
analytics into their respective teams because teams spend three-fourth of their revenue for paying salaries to the players
and to cover other expenditures. Hence, the teams cannot afford to invest huge sum of money in technology, data and analytical 
tools  \cite{davenport2014analytics}. This project would remove the above mentioned bottleneck and facilitate the teams to use 
analytical tools with much lower cost and at ease. 

\section{Earlier system}
Earlier to 2005, the data was collected by a person watching the game using either a notepad and pen or black boards with chalks.
This data was prone to human errors. As a result, the analysis carried out on this data and results were misleading.
In 2005, two Isareli scientists, Gal Oz and Miky Tamir, created a system called \textit{SportsVU} (see in figure \ref{fig:SportsVU})
\cite{mccann2012player} \cite{warsaw}. This
system captures the ball movement as well as athletes movement, all these data are combined together for statistial analysis using 
the statistical algorithms that the company has created \cite{warriors}.
Based on the statistial analysis inference, the players were chosen for a team, but this method was manual. Moreover, other data such 
as Rebounds, TurnOver and etc were calculated from this system as well as from manually gathered data.

\begin{figure}[H]
    \centering
    \includegraphics[scale=0.20]{STATS-SportVU-technology.png}
    \caption{SportsVU in basketball court.}
    \label{fig:SportsVU}
\end{figure}

\section{Background}
In this part, some general terms that are relative to the topic are going to be discussed. It is important to understand them for further
reading. 

The terms that will be discussed are:
\begin{itemize}
    \item  Draft picking and its process.
    \item  Machine learning.
    \item  Sports Analytics.
\end{itemize}
\subsection{Draft picking and its process}
NBA draft is an annual event where basketball teams select players from american colleges and from international professional league
for their rosters. Once a team selects a player, then the team has right to sign a NBA contract with the player.

In draft picking process, teams select eligible players in turns. There are two rounds in the draft where all 30 teams participate
to select a player in turns, meaning every year 60 eligible players are drafted, but teams that did not reach the playoffs 
in the previous regular season or teams with worst performs selects a player by undergoing a process called \textit{NBA Draft Lottery}.
This process determines the selection order of the team or provides an opportunity for the team which wins the lottery to pick the
first draft followed by other worst performing teams. The team with best records receives the 30\textsuperscript{th} pick. During the
second round in the draft, there is no lottery system, but teams pick the draft in the reverse order based on the previous regular 
season's standings. Moreover, the teams can exchange their draft picks with each other, for example, in 2019 the Minnesota Timberwolves 
traded the No. 11 pick and forward Dario Saric to the Phoenix Suns in exchange for the No. 6 pick. But, there are some restrictions
based on \textit{The Stepien Rule}, that is, this rule prevents the team from trading their first-round draft pick in consecutive 
years \cite{draft}.

\subsection{Machine learning}
Machine learning is a general concept and broader area which consists of many definitions provided by recognized and reliable 
universities, institutions, professors and organizations and they are as follows,
\begin{itemize}
    \item "Machine learning is based on algorithms that can learn from data without relying on rules-based programming." \cite{machineLearning1}
    \item "The field of Machine Learning seeks to answer the question “How can we build computer systems that automatically 
           improve with experience, and what are the fundamental laws that govern all learning processes?" \cite{machineLearning2}
    \item and etc.
\end{itemize}
\begin{figure}[H]
    \centering
    \includegraphics[scale=0.70]{Machine_Learning.png}
    \caption{Machine Learning in an eagle view}
    \label{fig:ml}
\end{figure}
Machine learning can be categorized in three types,
\begin{itemize}
    \item Supervised Learning.
    \item Unsupervised Learning.
    \item Reinforcement Learning.
\end{itemize}
The definitions for the above terms are:
\begin{itemize}
    \item "Supervised learning algorithms generate a function that maps inputs to desired outputs, based
    on a set of examples with known output (labeled examples)" \cite{tzanis2009machine}.
    \item "Unsupervised learning algorithms find patterns and relationships over a given set of inputs 
    (unlabeled examples)" \cite{tzanis2009machine}.
    \item "Reinforcement learning, where an algorithm learns a policy of how to act given an observation of the world" \cite{tzanis2009machine}.
\end{itemize}
In this project, we will mostly focus on Supervised and Unsupervised learning algorithms. The different types of algorithms in both
supervised and unsupervised are given below,

Some algorithms of supervised learning:
\begin{itemize}
    \item Nearest Neighbor
    \item Naive Bayes
    \item Support Vector Machine (SVM)
    \item Logistic Regression
    \item Linear Regression
    \item and etc.
\end{itemize}
Some algorithms of unsupervised learning:
\begin{itemize}
    \item k-means clustering
    \item Association Rules \cite{typeofml}
    \item and etc.
\end{itemize}
\subsection{Sports Analytics}
Sports analytics is the application of above mentioned algorithms to sport in order to draw useful insights which could help 
an individual athlete's performance, or a team's performance for a season. It can also help teams to perform injury analysis
and steps to mitigate them, salary of a player based on his previous performances and etc. Nowadays,  many teams, coaches and 
even players are adopting sports analytics for decision making.

"The analytics split nicely between the front-office and back-office. Front-office analytics
include topics like analyzing fan behavior, ranging from predictive models for season ticket
renewals and regular ticket sales, to scoring tweets by fans regarding the team, athletes,
coaches, and owners. This is very similar to traditional customer relationship management.
Financial analysis is also a key area, especially for the pros where salary caps or scholarship
limits are part of the equation. Back-office uses include analysis of both individual athletes as
well as team play. For individual players, there is a focus on recruitment models and scouting
analytics, analytics for strength and fitness as well as development, and predictive models for
avoiding overtraining and injuries. Concussion research is a hot field. Team analytics include
strategies and tactics, competitive assessments, and optimal roster choices under various onfield or on-court situations.” \cite{tichy2016changing}
\begin{figure}[H]
    \centering
    \includegraphics[scale=0.50]{sportsanalytics.jpg}
    \caption{Sports Analytics}
    \label{fig:sa}
\end{figure}
However, the analytical methods and data has to be kept safe and should be extremely careful because the data and methodology
could lead to numerous problems such as issues with betting companies, non-ethical training of atheletes leading to injuries and 
etc. \cite{spa}
\bibliographystyle{ACM-Reference-Format}
\bibliography{references}
\end{document}

